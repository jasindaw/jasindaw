\documentclass{article}
\usepackage{graphicx} % Required for inserting images
\usepackage[a4paper, left=3.00cm, right=3.00cm, bottom=3.00cm, top=2.00cm]{geometry}
\usepackage{authblk}
\usepackage[T1]{fontenc}
\usepackage{palatino}
\usepackage{enumitem}
\usepackage{mathtools}
\usepackage{amsmath} %bikin matriks
\usepackage{tabularray} %bikin matriks
\UseTblrLibrary{amsmath} %bikinmatriks
\usepackage{listings}
\usepackage{color}

\definecolor{dkgreen}{rgb}{0,0.6,0}
\definecolor{gray}{rgb}{0.5,0.5,0.5}
\definecolor{mauve}{rgb}{0.58,0,0.82}

\lstset{frame=tb,
  language=Matlab,
  aboveskip=3mm,
  belowskip=3mm,
  showstringspaces=false,
  columns=flexible,
  basicstyle={\ttfamily},
  numbers=none,
  numberstyle=\tiny\color{gray},
  keywordstyle=\color{blue},
  commentstyle=\color{dkgreen},
  stringstyle=\color{mauve},
  breaklines=true,
  breakatwhitespace=true,
  tabsize=3
}

\title{\textbf{Tugas Fisika Komputasi II: Optimasi Terkonstrain}}
\author{Jasinda Wijdannysa (5001201147)}
\affil{Departemen Fisika, Institut Teknologi Sepuluh Nopember}
\date{May 2023}

\begin{document}

\maketitle

\section*{Soal}
\begin{centering}
Diberikan fungsi sebagai berikut:
\[f=2x+3y-x^3-2y^2\]
dengan konstrain sebagai berikut:
\begin{enumerate}[label=(\alph*)] \centering
    \item \(x+3y-\frac{x^2}{2} \le \frac{11}{2}\)
    \item \(5x+2y+\frac{x^2}{10} \le 10\)
\end{enumerate}

untuk \(x \ge 0,\ y \ge 0\)

\end{centering}

\section*{Penyelesaian}
Konstrain yang awalnya berbentuk pertidaksamaan, perlu diubah menjadi persamaan dengan menambahkan variabel \(\theta_1^2\) pada konstrain (a) dan \(\theta_2^2\) pada konstrain (b). Maka, konstrain (a) dan (b) akan menjadi:

\begin{enumerate}[label=(\alph*)]
\centering
    \item \(\theta_1^2+x+3y-\frac{x^2}{2} - \frac{11}{2}=0\)
    \item \(\theta_2^2+5x+2y+\frac{x^2}{10} - 10=0\)
\end{enumerate}
Selanjutnya, konstrain yang telah diubah menjadi persamaan di atas akan digunakan untuk menuliskan persamaan berbentuk lagrangian dengan fungsi yang telah disebutkan sebelumnya.
\[\mathcal{L}=(2x+3y-x^3-2y^2)+\lambda_1(\theta_1^2+x+3y-\frac{x^2}{2} - \frac{11}{2})+\lambda_2(\theta_2^2+5x+2y+\frac{x^2}{10} - 10)=0\]
Terdapat 6 parameter pada persamaan lagrangian di atas, yaitu \(\ x,\ \ y,\ \lambda_1,\ \lambda_2,\ \theta_1,\ \) dan \(\ \theta_2\). Untuk mendapatkan nilai keenam parameter tersebut, turunan pertama dari \(\mathcal{L}\) terhadap masing-masing parameter harus bernilai 0.
\begin{equation}
    \frac{\partial \mathcal{L}}{\partial x}=(2-3x^2)+\lambda_1(1-x)+\lambda_2(5+\frac{x}{5})=0
\end{equation}
\begin{equation}
    \frac{\partial \mathcal{L}}{\partial y}=(3-4y)+\lambda_1(3)+\lambda_2(2)=0
\end{equation}
\begin{equation}
    \frac{\partial \mathcal{L}}{\partial \lambda_1}=\theta_1^2+x+3y-\frac{x^2}{2}-\frac{11}{2}=0
\end{equation}
\begin{equation}
    \frac{\partial \mathcal{L}}{\partial \lambda_2}=\theta_2^2+5x+2y+\frac{x^2}{10}-10=0
\end{equation}
\begin{equation}
    \frac{\partial \mathcal{L}}{\partial \theta_1}=2 \lambda_1 \theta_1=0
\end{equation}
\begin{equation}
    \frac{\partial \mathcal{L}}{\partial \theta_2}=2 \lambda_2 \theta_2=0
\end{equation}
\\
Dari keenam persamaan di atas, maka didapatkan 4 kasus untuk penyelesaian optimasi ini, yaitu sebagai berikut.

\begin{center}
\begin{tabular}{c|c}
     \textbf{Kasus 1}; \(\ \theta_1=\theta_2=0\) &  \textbf{Kasus 3}; \(\ \lambda_1=\theta_2=0\)\\
     \\
     \textbf{Kasus 2}; \(\ \theta_1=\lambda_2=0\) &  \textbf{Kasus 4}; \(\lambda_1=\lambda_2=0\)
\end{tabular}
\end{center}
\bigskip
Keempat kasus tersebut dapat dibuat matriks solusi sebagai berikut:
\begin{enumerate}
    \item \textbf{Kasus 1}
    \subitem 
    \begin{displaymath}
    f=
    \begin{pmatrix}
    (2-3x^2)+\lambda_1(1-x)+\lambda_2(5+\frac{x}{5})\\
    (3-4y)+\lambda_1(3)+\lambda_2(2)\\
    x+3y-\frac{x^2}{2}-\frac{11}{2}\\
    5x+2y+\frac{x^2}{10}-10
    \end{pmatrix}
    \begin{pmatrix}
        x\\y\\ \lambda_1 \\ \lambda_2
    \end{pmatrix}=
    \begin{pmatrix}
        0 \\ 0 \\ 0 \\ 0
    \end{pmatrix}
    \end{displaymath}
    
    \item \textbf{Kasus 2}
    \subitem 
    \begin{displaymath}
    f=
    \begin{pmatrix}
    (2-3x^2)+\lambda_1(1-x)\\
    (3-4y)+\lambda_1(3)\\
    x+3y-\frac{x^2}{2}-\frac{11}{2}\\
    \ \theta_2^2+5x+2y+\frac{x^2}{10}-10 \
    \end{pmatrix}
    \begin{pmatrix}
        x\\y\\ \lambda_1 \\ \lambda_2
    \end{pmatrix}=
    \begin{pmatrix}
        0 \\ 0 \\ 0 \\ 0
    \end{pmatrix}
    \end{displaymath}
    
    \item \textbf{Kasus 3}
    \subitem
    \begin{displaymath}
    f=
    \begin{pmatrix}
    (2-3x^2)+\lambda_2(5+\frac{x}{5})\\
    (3-4y)+\lambda_2(2)\\
    \theta_1^2+x+3y-\frac{x^2}{2}-\frac{11}{2}\\
    5x+2y+\frac{x^2}{10}-10
    \end{pmatrix}
    \begin{pmatrix}
        x\\y\\ \lambda_1 \\ \lambda_2
    \end{pmatrix}=
    \begin{pmatrix}
        0 \\ 0 \\ 0 \\ 0
    \end{pmatrix}
    \end{displaymath}
    
    \item \textbf{Kasus 4}
    \subitem
    \begin{displaymath}
    f=
    \begin{pmatrix}
    (2-3x^2)\\
    (3-4y)\\
    \theta_1^2+x+3y-\frac{x^2}{2}-\frac{11}{2}\\
    \theta_2^2+5x+2y+\frac{x^2}{10}-10
    \end{pmatrix}
    \begin{pmatrix}
        x\\y\\ \lambda_1 \\ \lambda_2
    \end{pmatrix}=
    \begin{pmatrix}
        0 \\ 0 \\ 0 \\ 0
    \end{pmatrix}
    \end{displaymath}
    
\end{enumerate}
\bigskip
Dari keempat matriks solusi tersebut, dapat dibuat sebuah \textit{function} pada Matlab yang berisi fungsi matriks solusi tersebut untuk keempat kasus.
\begin{lstlisting}
    function f=funcjob(X)
    %case 1
    %f=[(2-3*X(1)^2+X(3)*(1-X(1))+X(4)*(5+X(1)/5)); (3-4*X(2)+3*X(3)+2*X(4)); (X(1)+3*X(2)-X(1).^2/2-5.5); (5*X(1)+2*X(2)+X(1).^2/10-10)];

    %case 2
    %f=[(2-3*X(1)^2+X(3)*(1-X(1))); (3-4*X(2)+3*X(3)); (X(1)+3*X(2)-X(1).^2/2-5.5); (X(4)+5*X(1)+2*X(2)+X(1).^2/10-10)];

    %case 3
    %f=[(2-3*X(1)^2+X(4)*(5+X(1)/5)); (3-4*X(2)+2*X(4)); (X(3)+X(1)+3*X(2)-X(1).^2/2-5.5); (5*X(1)+2*X(2)+X(1).^2/10-10)];

    %case 4
    %f=[(2-3*X(1)^2); (3-4*X(2)); (X(3)+X(1)+3*X(2)-X(1).^2/2-5.5); (X(4)+5*X(1)+2*X(2)+X(1).^2/10-10)];
    end
\end{lstlisting}
\textit{Function} di atas dapat disimpan dengan ekstensi Matlab dengan nama "funcjob.m" dan \textit{function} ini dapat digunakan pada \textit{function} yang lain atau file Matlab yang lain untuk menjalankan suatu program.

\section*{Program pada Matlab}
Untuk menjalankan program dan mencari solusi parameter-parameter dari optimasi konstrain sebelumnya, maka dibuat sebuah program matlab sebagai berikut.

\begin{lstlisting}
clc; close all; clear all;
%initial
X(1,:)=[1 1 1 1]';
niter=30;tol=1e-6;

for ii=1:niter-1
    [f,dp]=jacobian(X(ii,:));
    dX=inv(dp)*f;
    X(ii+1,:)=X(ii,:)'-dX;
    fprintf('Iterasion=%i  solusi=%.4f  \n',ii,X(ii+1))
    if abs(X(ii+1,:)-X(ii,:))<tol
        r=X(ii+1,:);
        disp('Solusi Konvergence')
        break
    end
end

x=r(1); y=r(2); lambda_1=r(3); lambda_2=r(4);
f=2*x+3*y-x^3-2*y^2;
disp('Case 1')
disp(['x=' num2str(x)  ',y=' num2str(y), ',f=' num2str(f)])
disp(['lambda_1=' num2str(lambda_1), ',lambda_2=' num2str(lambda_2)])
\end{lstlisting}
\bigskip
Untuk menjalankan program di atas, diperlukan sebuah \textit{function} bernama "jacobian.m", yang dapat dituliskan pada program Matlab sebagai berikut.
\begin{lstlisting}
function [f0,jac]=jacobian(x)
h=1.0e-4;
n=length(x);
f0=funcobj(x);
jac=zeros(n,n);

for ii=1:n
    temp=x(ii);
    x(ii)=temp+h;
    f1=funcobj(x);
    x(ii)=temp;
    jac(:,ii)=(f1-f0)/h;
end
\end{lstlisting}
\textit{Function} jacobian ini akan berfungsi ketika \textit{function} funcjob telah dituliskan.

\section*{Hasil Run Program Matlab}
Untuk keempat kasus yang berbeda, maka kita dapat menggunakan persamaan f pada \textit{function} funcjob yang berbeda-beda pula sesuai dengan kasus yang ingin diselesaikan.

\subsection*{Kasus 1; \(\ \theta_1=\theta_2=0\)}
\begin{lstlisting}
Iterasion=1  solusi=1.3013  
Iterasion=2  solusi=1.2941  
Iterasion=3  solusi=1.2941  
Iterasion=4  solusi=1.2941  
Solusi Konvergence
Case 1
x=1.2941,y=1.6811,f=-0.18777
lambda_1=0.82726,lambda_2=0.62127
\end{lstlisting}
\subsection*{Kasus 2; \(\ \theta_1=\lambda_2=0\)}
\begin{lstlisting}
Iterasion=1  solusi=0.8571  
Iterasion=2  solusi=0.8526  
Iterasion=3  solusi=0.8526  
Iterasion=4  solusi=0.8526  
Solusi Konvergence
Case 2
x=0.85261,y=1.6703,f=0.51656
lambda_1=1.227,theta_2=2.3237
\end{lstlisting}
\subsection*{Kasus 3; \(\ \lambda_1=\theta_2=0\)}
\begin{lstlisting}
Iterasion=1  solusi=1.5079  
Iterasion=2  solusi=1.4826  
Iterasion=3  solusi=1.4826  
Iterasion=4  solusi=1.4826  
Solusi Konvergence
Case 3
x=1.4826,y=1.1837,f=0.45528
theta_1=1.5654,lambda_2=0.86737
\end{lstlisting}
\subsection*{Kasus 4; \(\lambda_1=\lambda_2=0\)}
\begin{lstlisting}
Iterasion=1  solusi=0.8333  
Iterasion=2  solusi=0.8167  
Iterasion=3  solusi=0.8165  
Iterasion=4  solusi=0.8165  
Solusi Konvergence
Case 4
x=0.8165,y=0.75,f=2.2137
theta_1=2.7668,theta_2=4.3509
\end{lstlisting}
\end{document}
